% Please do not change the document class
\documentclass{scrartcl}

% Please do not change these packages
\usepackage[hidelinks]{hyperref}
\usepackage[none]{hyphenat}
\usepackage{setspace}
\doublespace

% You may add additional packages here
\usepackage{amsmath}

% Please include a clear, concise, and descriptive title
\title{How to overcome different personality types in an agile team.}

% Please do not change the subtitle
\subtitle{COMP150 - Agile Development Practice}

% Please put your student number in the author field
\author{1706258}

\begin{document}

\maketitle

\abstract{The following paper will be an in depth look at how you can overcome personality types by using the steps that are given.}

\section{Introduction}
This paper will be on How to overcome different personality types in an agile team,This paper will go into detail about how you can take steps to overcome different personality types of individual members in a SCRUM team and how best to take on these situations that you can encounter when a team member has a different personality than another team member. This paper has three different research papers that the paper will go into detail.

\section{Your section title here}

A step to take to overcome different personality types would be to research into your own personality type and find out the traits that come with other peoples personality's as well as your own, one way to do this would be to look into the Myers-Briggs Type Indicator which is a tool that can be used to analyse a persons personality in reference [3] there is an explanation of the Myers-Briggs Type Indicator which reads "The Myers-Briggs Type Indicator MBTI)is a well-known instrument for measuring and understanding individual personality types."

In the Myers-Briggs Type Indicator, there are 4 sets of preferences extroversion (E) and Introversion (I), Sensing (S) and Intuition (N), Thinking (T) and Feeling (F), Judging (J) and Perceiving (P) A persons Personality is identified through these four sets, in reference [1] The writer has a reference from "Manual: A Guide to the Development and Use of the Myers-Briggs Type Indicator," which reads "Their studies suggest that individuals are expected to possess one of four preferences in their behaviour. For a person's energetic preference: they are either extrovert or introvert; for what they perceive: sensing or intuition; for their decisions: thinking or feeling, and their lifestyle: judging or perceiving.", each person will select one trait from each pair to figure out what there own personality type is, in which there are 16 possible types, In reference [3] there is a table which lists all 16 personality outcomes. 

Traits that are part of anyone's personality are Extroversion (E) and Introversion (I) reference [2] has an explanation for both traits which reads "Introversion (I): They are usually involved with ideas; they prefer to reflect before acting. They need time to think and recover your energy. In general, they are little sociable." and "Extroversion (E): Usually act; like to perform various activities; act first and then think. When inactive, your energy decreases. In general, they are sociable and not with the emotions." reference [3] also has a short explanation "While Es prefer looking outward, Is have an inward view. Es are talkative, outgoing,	conversation initiators. Is, in contrast, are quiet, reserved, and tend to respond to conversation rather than start it." After (E) and (I) is Sensing (S) which is when the person relies more on facts and statistics because they need more information and prefer to have prior knowledge of the subject, and Intuition (N) is when the person prefers exploring into a subject and looking for an alternative explanation to a problem.

The next two traits are Thinking (T) which is when the person acts on logic and are rational and prefer formal methods and Feeling (F) which is when someone acts on their feelings and lets it affect their views and values.Judging (J) people who are judgers like to have things pre-planned and in order alternatively Perceivers (P) prefer to make spontaneous decisions and appear to be disorganized.

Using this indicator you should be able to pick up on other team members personality traits and figure out what type of personality they have, firstly you can find out what your personality is and then infer what your team members personalities are and spot which kind of personalities conflict with each other when working in a team, for example someone with the personality trait of Judging (J) who likes things to be preplanned and for everything to be organized and another team member that has the trait Percievence (P) which is often unorganized and makes decisions spontaneously, these two traits can conlifct when the person that has trait (J) sees that the team member with the trait (P) is unorganized and they dislike the fact that the person is unorganized and if this is not identified and spoken about it can get to boiling point when team members are falling out and start to dislike each other but if it is identified and spoken about with both members of the team  they can then work out how best to deal with each other's personality traits. 


\section{Conclusion}

In conclusion, this paper has gone into detail on the steps you can take into consideration to use when you are in a SCRUM team, these steps should help in identifying problems through different personality types and solving these problems before they reach boiling point.

\bibliographystyle{ieeetran}
\bibliography{references}

\end{document}
